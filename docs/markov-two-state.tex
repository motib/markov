% !TeX root = markov.tex

\section{The two-state process}\label{s.two-state}

The two-state process is similar to the Ehrenfest model in that the probabilities at each step are different and we are interested in the stationary probability distribution of the unbounded process. There are two states $A,B$. In state $A$ the process transitions to $B$  with probability $a$ and remains in $A$ with probability $1-a$. Similarly, the probability of a transition from $B$ to $A$ is $b$ and the probability of remaining in $B$ is $1-b$.
\begin{center}
\begin{tikzpicture}[->,node distance = 6mm and 2cm]
\node[draw,circle,minimum size=10mm] (A) {$A$};
\node[draw,circle,minimum size=10mm] (B) [right=of A] {$B$};
\draw (A) edge[bend left] node[above] {$a$} (B);
\draw (B) edge[bend left] node[below] {$b$} (A);
\draw (A) edge [loop left] node {$1-a$} (A);
\draw (B) edge [loop right] node {$1-b$} (B);
\end{tikzpicture}
\end{center}
The stationary distribution, that is, the proportion of visits to $A$ and to $B$ is:
\[
\left[\frac{b}{a+b}, \frac{a}{a+b}\right]\,.
\]
Here is an output of a simulation:
\begin{verbatim}
Probabilities:  a = 0.500, b = 0.333
Theoretical stationary distribution: A = 0.400, B = 0.600
Simulation  stationary distribution: A = 0.402, B = 0.598
\end{verbatim}
When $a+b=1$ the probability of being at $A$ is $b$ and the probability of being at $B$ is $a$:
\begin{verbatim}
Probabilities:  a = 0.333, b = 0.667
Theoretical stationary distribution: A = 0.667, B = 0.333
Simulation  stationary distribution: A = 0.674, B = 0.326
\end{verbatim}
You can enter a required proportion $p$ of visits to $B$ and any probability $0<a<p$. The proportion will be achieved for:
\[
b = \frac{a(1-p)}{p}\,,
\]
as shown in the following simulation where we entered $p=0.8, a=0.6$:
\begin{verbatim}
Probabilities:  a = 0.600, b = 0.150, proportion = 0.800
Theoretical stationary distribution: A = 0.200, B = 0.800
Simulation  stationary distribution: A = 0.194, B = 0.806
\end{verbatim}
