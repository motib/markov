% !TeX root = prob.tex

%%%%%%%%%%%%%%%%%%%%%%%%%%%%%%%%%%%%%%%%%%%%%%%%%%%%%%%%%%%%%%%%

\thispagestyle{empty}

\begin{center}
\textbf{\LARGE Markov Chain Simulations}

\bigskip
\bigskip
\bigskip

\textbf{\Large Moti Ben-Ari}

\bigskip

\url{http://www.weizmann.ac.il/sci-tea/benari/}

\bigskip
\bigskip
\bigskip

%Version $1.1$
%
%\bigskip

\today

\end{center}

\vfill

\begin{center}
\copyright{} Moti Ben-Ari $2023$
 \end{center}
 
\begin{small}
This work is licensed under Attribution-ShareAlike 4.0 International. To view a copy of this license, visit \url{http://creativecommons.org/licenses/by-sa/4.0/}.
\end{small}
\newpage

\tableofcontents

\newpage

\begin{center}
\textbf{\LARGE Introduction}
\end{center}

\addcontentsline{toc}{section}{\large Introduction}

\bigskip

Simulations are an excellent way of understanding probability, especially, the behavior of processes of long duration. The programs enable the user to perform experiments by varying the parameters of problems interactively and analyzing the results, both printed and displayed in graphs. The simulations are of processes known as \emph{Markov chains}, where the next state of the system depends only on the current state and not on the history of how the process got to the current state. 

The definitions of the problems and their theoretical solutions are taken from the references at the end of the document.

\nocite{*}

The following problems are simulated: the \emph{gambler's ruin} in 
Section~\ref{s.gamblers}, one-, two- and three-dimensional \emph{random walk} in Section~\ref{s.walk}, the \emph{Ehrenfest model} and the \emph{two-state process} in Section~\ref{s.ehrenfest}.

\subsection*{Technical notes}

The programs are written in the Python 3 language and use the \verb+matplotlib+ library to generate the graphs. Parameters directly related to the problems, such as the probability of success, can be modified interactively. Others, related to the simulation, such as the number of steps in a simulation, are defined in a module \verb+configuration.py+ containing just declarations of values that are easy to modify.

You need to install the Python (\url{https://www.python.org/downloads/}) although a knowledge of Python programming is not necessary.

To run in the IDLE or Thonny environments, change the configuration constant \verb+CLOSE+ to \verb+True+. When the simulation is run multiple times, you will have to close each figure before running a new simulation. This is not necessary if the programs are run in Visual Studio Code or from the command line.
