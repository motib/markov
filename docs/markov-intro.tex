% !TeX root = markov.tex

\thispagestyle{empty}

\begin{center}
\textbf{\LARGE Markov Chain Simulations}

\bigskip
\bigskip
\bigskip

\textbf{\Large Moti Ben-Ari}

\bigskip

\url{http://www.weizmann.ac.il/sci-tea/benari/}

\bigskip
\bigskip
\bigskip

\today

\end{center}

\vfill

\begin{center}
\copyright{} Moti Ben-Ari $2023$
 \end{center}
 
\begin{small}
This work is licensed under Attribution-ShareAlike 4.0 International. To view a copy of this license, visit \url{http://creativecommons.org/licenses/by-sa/4.0/}.
\end{small}
\newpage

\tableofcontents

\newpage

\section{Introduction}

Simulations are an excellent way of understanding probability, especially, the behavior of processes of long duration. Simulation programs enable the user to perform experiments by varying the parameters of problems interactively and analyzing the results, both printed and displayed in graphs. The simulations in this archive are of processes known as \emph{Markov chains}, where the next state of the system depends only on the current state and not on the history of how the process got to the current state. 

The following problems are simulated: the \emph{gambler's ruin} in 
Section~\ref{s.gamblers}, one-, two- and three-dimensional \emph{random walk} in Section~\ref{s.walk}, the \emph{Ehrenfest model} in Section~\ref{s.ehrenfest} and the \emph{two-state process} in Section~\ref{s.two-state}.

\subsection*{Technical notes}

The programs are written in the Python 3 language and use the \verb+matplotlib+ library to generate the graphs. You need to install Python (\url{https://www.python.org/downloads/}) although a knowledge of Python programming is not necessary to run the simulations.

To run in the IDLE or Thonny environments, change the configuration constant \verb+CLOSE+ to \verb+True+. When the simulation is run multiple times, you will have to close each figure before running a new simulation. This is not necessary if the programs are run in the Visual Studio Code environment or from the command line.

Parameters directly related to the problems, such as the probability of success, can be modified interactively. Others, related to the simulation, such as the number of steps in a simulation, are defined in a module \verb+configuration.py+ which just contains declarations of values so it is easy to modify. The code that uses \verb+matplotlib+ appears in a separate module.

\subsection*{Sources}

A knowledge of probability is assumed at the level of the first few chapters of \cite{BW,ross}. These textbooks include examples of the gambler's ruin and random walk, as well as short presentations of Markov chains. The clearest explanational of one-dimensional random walk is in \cite{border}. I was introduced to these problems by Mosteller \cite{mosteller}; my ``reworking'' of this book is in \cite{mos}. A comprehensive work on Markov chains is \cite{privault} from which I took most of the theoretical results.
