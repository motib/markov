% !TeX root = prob.tex

\section{Gambler's Ruin}\label{s.gamblers}

\textbf{Problem} Two players $A$ and $B$ compete in a contest. There is an initial finite capital of $n$ units: $A$ has $i$ and $B$ has $n-i$. They repeated play a game such that the probability that $A$ wins is $p$ and the probability that $B$ wins is $q=1-p$. The loser gives one unit to the winner. When one player has $n$ units the contest is finished.
\begin{enumerate}
\item Given initial parameters $(p, n, i)$, what is the probability that $A$ wins?
\item What is the expectation of the duration of the game?
\end{enumerate}
%\begin{figure}[tb]
\begin{center}
\begin{tikzpicture}[scale=1.2]
\draw (0,0) node[above left] {$A$} -- 
      (10,0) node[above right] {$B$};
\foreach \x in {0,1,2,3,4,5,6,7,8,9,10} {
  \draw (\x,0) -- +(0,4pt);
  \node at (\x,-10pt) { $\x$ };
}
\node at (4,-9mm) {$i$};
\node at (10,-9mm) {$n$};
\draw[fill] (4,7mm) circle[radius=1pt];
\draw[->] (4,7mm) -- node[above] {$q$} +(-1,0);
\draw[->] (4,7mm) -- node[above] {$p$} +(1,0);
\end{tikzpicture}
\end{center}
%\caption{What is the probability that the particle reaches $0$ or $n$?}\label{f.ruin3}
%\end{figure}
The gambler's ruin is presented in most books on probability such as \cite[Section~2.7.2]{BW}, \cite[Section~3.4]{ross},\cite{mosteller}, \cite{mos}. \cite[Chapter~2]{privault} has a more extensive discussion which includes the solution to the expectation of the duration of the contest.\footnote{Privault presentation asks for the probability that $A$ is ruined, that is, that $B$ wins. I follow other references who ask for $A$'s probability of winning.}

\subsection{Theoretical solutions}

Given $(p,n,i)$ the probability that $A$ wins the contest is:
\[
P_A(p, n, i) = \left(\frac{1-r^{i}}{1-r^n}\right)\,,
\]
where $r=q/p$. By symmetry, the probability that $B$ wins is:
\[
P_B(p, n, i) = \left(\frac{1-(1/r)^{n-i}}{1-(1/r)^{n}}\right)\,.
\]

There are separate solutions for $p\neq 1/2$ and $p=1/2$. For $p\neq 1/2$ the expectation of the duration of the contest is:
\[
P_{\mathit{duration}}(p,n,i)=\frac{1}{q-p}\left(i-n
\frac{1-r^k}{1-r^n}\right)\,.
\]
For $p=1/2$ the expectation of the duration of the contest is:
\[
P_{\mathit{duration}}(p,n,i)=i(n-1)\,.
\]
Of course the duration does not depend on which player wins. If $A$ wins, the contest terminates for $B$ also, and conversely.

\subsection{Program structure}

The program is divided into three modules:
\begin{itemize}
\item \verb+configuration.py+ contains declarations of variables which are intended to be constants.
\item \verb+gambler_plot.py+ contains the functions for plotting the histogram of the duration of the contests. If the simulation is run for multiple probabilities or initial values, a graph of the probability of wins is also displayed. The module imports \verb+matplotlib.pyplot+.
\item \verb+gamblers_ruin.py+ is the main program which obtains the parameters, runs the simulations and prints the output.
\end{itemize}

\subsection{Running the simulations}

The program runs the simulations in a loop, each time asking the user how to run it. You can run the same simulation again with the saved parameters, enter new parameters, or run a sequence of simulations for a range of probabilities or initial values.

A typical output is as follows:
\begin{verbatim}
Probability = 0.450, capital = 20, initial = 10
Wins = 127, losses = 873, limits exceeded = 0
Proportion of wins     = 0.1270
Probability of winning = 0.1185
Average duration  = 78
Expected duration = 76
\end{verbatim}
The results of the simulation are very close to the theoretical probability and duration.

Figure~\ref{f.gambler-hist1} shows the histogram for the duration of the contest, limited to a duration of $200$. The vertical line is the expectation.
\begin{figure}
\begin{center}
\includegraphics[width=\textwidth]{gamblers_ruin-01}
\end{center}
\caption{Histogram for $p=0.45, n=20, i=10$}\label{f.gambler-hist1}
\end{figure}

Figure~\ref{f.gambler-hist2} proportion of wins for multiple probabilities. It also shows histograms for the duration of the contest for these probabilities. 
\begin{figure}
\begin{center}
\includegraphics[width=\textwidth]{gamblers_ruin-02}
\end{center}
\caption{Proportion of wins and histogram for $n=20, i=10$ and multiple probabilities}\label{f.gambler-hist2}
\end{figure}
